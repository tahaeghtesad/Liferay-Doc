\section{Persian Support and Internationalization}
If you would like to make the Liferay 7 theme RTL and use Persian in the UI, follow these steps:
\begin{enumerate}
    \item Goto Control Panel
    \item Goto Configuration
    \item Goto Instance Settings
    \item Goto Miscellaneous tab
    \item Set default language as Persian
    \item Click on save
\end{enumerate}

\subsection{I18n}
For internationalizing your portlet based on the selected language of a user (or default language, do the following.

In \textit{resources/content} create a file with the name of \textit{\textbf{Language\textunderscore fr.properties}}. The \textit{fr} keyword here means that this language would be for the users who choose French\textit{(fr\textunderscore FR)} as their language.

List of supported languages are as follows:

\lstset{language=bash}
\begin{lstlisting}[caption=Supported Languages]
locales=ar_SA,eu_ES,bg_BG,ca_AD,ca_ES,zh_CN,zh_TW,hr_HR,cs_CZ,da_DK,nl_NL,
    nl_BE,en_US,en_GB,en_AU,et_EE,fi_FI,fr_FR,fr_CA,gl_ES,de_DE,el_GR,
    iw_IL,hi_IN,hu_HU,in_ID,it_IT,ja_JP,ko_KR,lo_LA,lt_LT,nb_NO,fa_IR,
    pl_PL,pt_BR,pt_PT,ro_RO,ru_RU,sr_RS,sr_RS_latin,sl_SI,sk_SK,es_ES,
    sv_SE,tr_TR,uk_UA,vi_VN
\end{lstlisting}

The file with no suffix, means that it will be loaded when the selected language is not in the list of provided languages. (In most cases, it is in English)

A sample for internationalization would be like this:

\lstset{language=jsp}
\begin{lstlisting}[caption=Language.properties]
javax.portlet.display-name.awesome-portlet=awesome-portlet JSP
javax.portlet.keywords.awesome-portlet=awesome-portlet,jsp
javax.portlet.short-title.awesome-portlet=awesome-portlet JSP
javax.portlet.title.awesome-portlet=awesome-portlet JSP Portlet
awesome-portlet.caption=Hello from awesome-portlet JSP!
\end{lstlisting}

\lstset{language=jsp}
\begin{lstlisting}[caption=Language\_fr.properties]
javax.portlet.display-name.awesome-portlet=impressionnant portlet prénom
javax.portlet.short-title.awesome-portlet=impressionnant portlet titre court
javax.portlet.keywords.awesome-portlet=impressionnant portlet
javax.portlet.title.awesome-portlet=impressionnant portlet titre
awesome-portlet.caption=bonjour le monde impressionnant portlet
\end{lstlisting}

For adding one of these keys into your JSP files, use the \textit{message} element of \textit{liferay-ui} taglib.

\lstset{language=jsp}
\begin{lstlisting}[caption=view.jsp]
<p>
	<b><liferay-ui:message key="awesome-portlet.caption"/></b>
</p>
\end{lstlisting}

This will automatically use the key defined for the specific language which the user has chosen.

\subsubsection{I18n in REST}

For using internationalization in REST applications in Liferay, add the same bundles mentioned above, then use \textit{HTTPServletRequest}'s \textit{getLocale()} method to resolve the required key:


\lstset{language=java}
\begin{lstlisting}[caption=RestApplication.java]
@GET
@Path("/customMessage")
public String customMessage(@Context HttpServletRequest request){
	return ResourceBundle.getBundle("content.Language", request.getLocale()).getString("my-key");
}
\end{lstlisting}