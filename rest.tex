\section{RESTful Services in Liferay}

Liferay can provide web services using multiple frameworks like \href{https://docs.spring.io/spring/docs/current/spring-framework-reference/web.html}{Spring-MVC}, \href{https://en.wikipedia.org/wiki/Apache_CXF}{Apache CXF} and etc. While the default one is Apache CXF. Apache CXF can be extended using \href{https://en.wikipedia.org/wiki/Java_API_for_RESTful_Web_Services}{JAX-RS} and \href{https://en.wikipedia.org/wiki/Java_API_for_XML_Web_Services}{JAX-WS}.

\subsection{JAX-RS}
Original Tutorial: \url{https://dev.liferay.com/develop/tutorials/-/knowledge_base/7-0/jax-ws-and-jax-rs}

Providing RESTful web services are as easy as a popsicle in Liferay using J2EE. You create your service class by extending \textit{javax.ws.rs.core.Application}. Provide an application path by \textbf{@ApplicationPath} annotation. Override \textit{getSingletons()} and provide \textit{this} instance as an application component. And deploy it with OSGI \textbf{@Component} annotation.

Serialization to JSON objects can be done via Jackson. A wrapper called \textit{com.fasterxml.jackson.jaxrs.json.JacksonJsonProvider} is available for Jackson to provide serialization and deserialization to JAX-RS. It can be found in the following repository:

\lstset{language=groovy}
\begin{minipage}{\linewidth}
\begin{lstlisting}
compileInclude group: 'com.fasterxml.jackson.jaxrs', name: 'jackson-jaxrs-json-provider', version: '2.9.2'
\end{lstlisting}
\end{minipage}

\textbf{It needs to be added as \textit{compileInclude} because liferay doesn't support it internally}. Also, don't forget to add an instance as singleton in \textit{getSingletons()} method.

Ok! Now the whole source code to our \textit{AwesomeRest} can be like this:

\lstset{language=Java}
\begin{lstlisting}[caption=AwesomeRest.java]
package ir.hamisystem.liferay;

import com.fasterxml.jackson.jaxrs.json.JacksonJsonProvider;
import ir.hamisystem.liferay.service.SurveyLocalService;
import org.osgi.service.component.annotations.Component;
import org.osgi.service.component.annotations.Reference;

import javax.ws.rs.ApplicationPath;
import javax.ws.rs.GET;
import javax.ws.rs.Path;
import javax.ws.rs.Produces;
import javax.ws.rs.core.Application;
import java.util.HashSet;
import java.util.Set;

@Component(immediate = true, service = Application.class)
@ApplicationPath("/AwesomeRest")
public class AwesomeRest extends Application{

    private SurveyLocalService surveyLocalService;

    @Reference(bind = "-")
    public void setSurveyLocalService(SurveyLocalService surveyLocalService) {
        this.surveyLocalService = surveyLocalService;
    }

    @Override
    public Set<Object> getSingletons() {
        HashSet<Object> objects = new HashSet<>();
        objects.add(new JacksonJsonProvider());
        objects.add(this);
        return objects;
    }

    @GET
    @Path("/hello")
    @Produces("text/html")
    public String sayHello() {
        return "Hello";
    }

    @GET
    @Path("/latestSurvey")
    @Produces("application/json")
    public Object getLatestSurvey() {
        return surveyLocalService.getSurveies(-1,-1);
    }


}
\end{lstlisting}


Then using your Liferay Admin panel, navigate to  Configuration → System Settings → Foundation → CXF Endpoints. Click on the + button and provide a \textbf{Context Path}

After that navigate to Configuration → System Settings → Foundation → REST extenders. Click on the + button and expand your CXF endpoint. \textit{(both these endpoints must be the same)}

Ok. Now you can make a call to \url{http://localhost:8080/o/<endpoint_name>/AwesomeRest/hello} or if you navigate to \url{http://localhost:8080/o/<endpoint_name>/} all registered services will be shown.